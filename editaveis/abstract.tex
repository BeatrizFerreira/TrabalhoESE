\begin{resumo}[Abstract]
 \begin{otherlanguage*}{english}
Software architecture can be defined as a software development fundamental activity: it is during this activity that, once understood, problems and needs the software ought to satisfy are designed. In other words, it is over the software architecture activity that the software elements are identified and related to the accomplishment of a successful and well done code implementation. Well-designed software architecture allows insertions, changes and removal of functionalities from/to software systems without much effort, making the maintainability of the system easier. The agile methodology has as one of its principles accepting changes requests and new functionalities, dealing with constant software mutability. Such changes cause software architecture changes, and, when those changes are uncontrolled, the software development and maintenance might be lead to chaos. This research aims to indicate possible ways that constant changes during software development can become a matter of concerns related to the maintenance phase and attributes inherent to systems maintainability.

   \vspace{\onelineskip}
 
   \noindent 
   \textbf{Key-words}: 1. Software Architecture. 2. Agile Methodology. 3. Maintainability. 4. Software Engineering.
 \end{otherlanguage*}
\end{resumo}
