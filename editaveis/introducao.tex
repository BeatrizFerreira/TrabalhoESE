\chapter[Introdução]{Introdução}
%\addcontentsline{toc}{chapter}{Introdução}

O processo de desenvolvimento de \textit{software}, segundo Sommerville (2003), compreende, independentemente do modelo de processo adotado, as fases de especificação, projeto e implementação, validação e manutenção e evolução de software: após a definição de funcionalidades e restrições do sistema na fase de especificação, a modelagem e implementação do sistema é realizado de acordo com os dados sobre as funcionalidades e restrições definidos na fase de especificação do sistema; uma vez implementado, o sistema deverá ser validado (fase de validação) e posteriormente mantido e evoluído conforme a demanda (SOMMERVILLE, 2004). 

Dentre as atividades de modelagem temos o desenvolvimento da arquitetura de \textit{software}, geralmente realizada entre atividades de compreensão do problema ou necessidade para o qual o \textit{software} será construído (especificação) e a sua implementação. Arquitetura de \textit{software} é definida como “a estrutura ou estruturas de sistema que comprimem elementos de \textit{software}, as propriedades visivelmente externas de tais elementos e o relacionamento entre eles” (BASS, CLEMENTS, KASMAN, 2003).

Além de modelos de processo, o desenvolvimento de \textit{software} é executado adotando-se uma metodologia de desenvolvimento. No princípio, foi adotada a metodologia tradicional para desenvolvimento, que procura planejar e documentar todo o \textit{software} para posterior implementação, processo geralmente aplicado à produção de produtos visíveis e tangíveis, como automóveis e aeroplanos (PRESSMAN, 2006). Sendo os produtos de \textit{software} descritos por Brooks (1987) como invisíveis e por isto mutáveis e complexos, que devem estar em conformidade com requisitos pré-estabelecidos pelo usuário ou cliente, à necessidade de uma metodologia de desenvolvimento que melhor se adequasse a estas características foi proposta no ano de 2001 a partir do Manifesto Ágil, escrito por Kent Beck e outros 16 estudiosos, uma nova metodologia a ser aplicada ao desenvolvimento de \textit{software}: a metodologia ágil.

Por ser um produto invisível e intangível, há uma dificuldade de idealizar tanto o processo quanto o produto de software como coisas reais e tangíveis e, portanto, a mutabilidade se destaca como uma característica onipresente em produtos de \textit{software} (BROOKS, 1987; OSTERWEILL, 1987). A metodologia ágil visa “satisfazer o cliente desde o início por meio da entrega contínua de \textit{software} valioso” e trata a mutabilidade do \textit{software} como algo natural, uma vez que é uma propriedade essencial do produto invisível de \textit{software}, e aceitando todas e quaisquer mudanças, mesmo em momentos tardios do desenvolvimento (PRESSMAN, 2006).

Mudanças no \textit{software} significam mudanças na arquitetura do mesmo, que deve estar preparada para receber tais mudanças e é um elemento que deve ser sempre considerado ao aprovar mudanças sem análise prévia de seus impactos. Tais mudanças na arquitetura do \textit{software} afetam a manutenibilidade do \textit{software},  seja tal modificação feita para adaptação, correção ou para atender requisições do usuário (ROMBACH, 1990). Entenda por manutenibilidade a definição feita pela ISO/IEC 25010/2011 como  “o grau de eficácia e eficiência com o qual um produto ou sistema pode ser modificado”.

A presente pesquisa visa responder à questão “\textit{Como as mudanças na arquitetura de um software impactam na manutenibilidade do sistema quando utilizada a metodologia ágil para o desenvolvimento?}”, visto que a metodologia ágil prega a adaptabilidade do sistema para quaisquer modificações requisitadas que significam mudanças na arquitetura do \textit{software}. Quando a arquitetura do sistema é desenvolvida de maneira a ser suficientemente adaptativa a tais mudanças, pode-se presumir que a manutenção a feita no \textit{software} após sua implantação pode ser realizada sem grandes dificuldades.

A resposta que deseja-se encontrar durante a execução deste trabalho tem como objetivo central apontar de que forma estas mudanças na arquitetura afetam a manutenibilidade do \textit{software} quando adotada a metodologia ágil para o desenvolvimento do mesmo.

Ao indicar tais impactos, a sociedade envolvida no desenvolvimento e qualidade dos produtos de \textit{software} poderá voltar-se aos problemas encontrados, pesquisando por soluções adequadas que os resolvam ou apontando soluções já existentes.

\section{Organização}
Este trabalho foi organizado de modo a proporcionar ao leitor as principais ideias a respeito dos objetos tratados por este estudo, que será dividido em quatro diferentes partes. A primeira parte da pesquisa realizada tratará do conceito de arquitetura de \textit{software} bem como os princípios definidos pela literatura selecionada para a construção de uma arquitetura e design ótimo de um \textit{software}.

Durante a segunda parte será abordada a metodologia ágil e os princípios ágeis de desenvolvimento, com enfoque maior no princípio de aceitação de mudanças a qualquer momento do desenvolvimento.
Na penúltima parte da pesquisa, serão abordados os conceitos de manutenibilidade e manutenção de \textit{software}. 

Por fim, na última parte deste trabalho, serão tratadas as considerações finais sobre a conclusão da revisão sistemática realizada, visando responder parte da questão aqui proposta para investigação.

A metodologia adotada com todos os passos para a execução da pesquisa pode ser encontrada no capítulo final deste trabalho.