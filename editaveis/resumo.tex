\begin{resumo}
A arquitetura de software pode ser dita como uma atividade de fundamental importância na construção de um software: é durante esta atividade que, uma vez entendidos, os problemas e necessidades que o software atenderá serão modelados, ou seja, os elementos contidos no software serão identificados e relacionados para que a atividade posterior, de implementação, possa ser realizada com sucesso. Uma boa arquitetura permite a inserção, modificação ou retirada de funcionalidades de um sistema de software sem que seja necessário despender-se de enorme esforço, ou seja, facilita a manutenibilidade do sistema. A metodologia ágil tem como um de seus princípios a aceitação de requisições de mudanças e adição de novas funcionalidades e procura tratar a mutabilidade constante do software. Tais mudanças no software acarretam em mudança na arquitetura do mesmo e, quando não controladas, podem levar o desenvolvimento e manutenção do software ao caos. A presente pesquisa visa apontar possíveis maneiras de como estas modificações constantes durante o desenvolvimento de software pode afetar a fase de manutenção e as subcaracterísticas inerentes à característica de manutenibilidade do sistema.

 \vspace{\onelineskip}
    
 \noindent
 \textbf{Palavras-chaves}: 1. Arquitetura de Software. 2. Metodologia Ágil. 3. Manutenibilidade 4. Engenharia de Software. 
\end{resumo}
