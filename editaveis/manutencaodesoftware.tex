\chapter[Manutenção de Software]{Manutenção de Software}
Segundo Sommerville (2004), manutenção de software “é o processo de modificação de um sistema depois que este foi colocado em uso”. Os estudos realizados por Lehman e Belady (1985) sobre mudanças do sistema propõem um conjunto de leis sobre o assunto, conhecidas como Leis de Lehman. A lei de maior relevância para o tópico aqui discutido é a primeira lei, que define a manutenção de \textit{software} como um processo inevitável, pois ao passo que determinado sistema é implantado no ambiente destinado e este ambiente modifica-se, a necessidade de mudança no sistema também ocorre.

Para que tal processo ocorra, existem diferentes estratégias cada qual com um objetivo distinto. Estas estratégias são manutenção de \textit{software}, transformação da arquitetura e reengenharia de \textit{software} (SOMMERVILLE, 2004).

Sommerville (2004) cita além das estratégias que podem ser seguidas para execução doprocesso de manutenção de software, este também possui três classificações distintas:
\begin{itemize}
	\item \textbf{Manutenção corretiva:} visa identificar e corrigir erros encontrados no sistema.
	\item \textbf{Manutenção adaptativa:} procura adaptar o sistema a um novo ambiente operacional.
	\item \textbf{Manutenção perfectiva:} tem como objetivo atender à solicitação mudanças de funções existentes ou inclusão de mudanças existentes, bem como melhorar o \textit{software} de uma maneira geral.

\end{itemize}

\section{Manutenibilidade}
Como anteriormente citado, a norma SQuARE (ISO/IEC 25010/2011) define manutenibilidade como  “o grau de eficácia e eficiência com o qual um produto ou sistema pode ser modificado”. Este modelo de qualidade de produto de software divide esta característica em outras cinco subcaracterísticas:

\begin{itemize}

	\item \textbf{Modularidade:} grau com o qual um sistema é composto por diferentes módulos interrelacionados de modo que qualquer mudança em um destes módulos impacta na mudança de outro(s) módulo(s) (ISO/IEC 25010/2011).
	\item \textbf{Reusabilidade:} grau com o qual um recurso do sistema pode ser utilizado em outros sistemas (ISO/IEC 25010/2011).
	\item \textbf{Anasabilidade:} grau de efetividade e eficiência com o qual é possível identificar os impactos de uma mudança no sistema, suas deficiências ou causas de falhas e partes a serem modificadas (ISO/IEC 25010/2011).
	\item \textbf{Modificabilidade:} grau com o qual um sistema pode ser modificado sem que novos defeitos sejam introduzidos (ISO/IEC 25010/2011).
	\item \textbf{Testabilidade:} grau de efetividade e eficiência com o qual os critérios de teste podem ser estabelecidos e executados (ISO/IEC 25010/2011).

\end{itemize}

	Dentre os diversos fatores existentes que podem afetar a manutenibilidade do \textit{software} podemos destacar a arquitetura, a documentação, tecnologias empregadas no desenvolvimento e a compreensibilidade do programa (BRUSAMOLIN, 2004).
