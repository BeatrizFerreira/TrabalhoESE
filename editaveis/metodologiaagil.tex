\chapter[Metodologia Ágil]{Metodologia Ágil}
Também chamada de metodologia “leve” pelos próprios criadores, a metodologia ágil não se declara “anti-metodológica” e procura estabelecer um balanceamento entre a confecção de documentação e o planejamento pesado da produção de \textit{software} e a produção de \textit{software} de fato. O “Manifesto Ágil para o desenvolvimento de \textit{Software} Ágil”, assinado em 2001 por Kent Beck  e outros 16 produtores, desenvolvedores e consultores de \textit{software}, declara as seguintes palavras:
\begin{quotation}
“Estamos descobrindo maneiras melhores de desenvolver \textit{software} fazendo-o nós mesmos e ajudando outros a fazê-lo. Através deste trabalho, passamos a valorizar:
	\textbf{Indivíduos e interação entre eles} mais que processos e ferramentas.
	\textbf{\textit{Software} em funcionamento} mais que documentação abrangente.
	\textbf{Colaboração com o cliente} mais que negociação de contratos.
	\textbf{Responder a mudanças} mais que seguir um plano.

Ou seja, mesmo havendo valor nos itens à direita, valorizamos mais os itens à esquerda.”
\end{quotation}

Sendo mais facilmente aplicável à pequenas e médias organizações, a metodologia ágil mais difundida é a \textit{Extreme Programming}, que se baseia em requisitos simples que se modificam de forma rápida (BECK, 1999). Segundo BECK (1999), os quatro valores que norteiam esta metodologia são comunicação, simplicidade, \textit{feedback} e coragem. Estes quatro valores baseiam-se nas ideias de que a comunicação entre membros da equipe e entre equipe e clientes deve ser efetiva e feita pessoalmente sempre que possível, sendo o principal fator para o sucesso do projeto. A simplicidade diz respeito à implementação da solução: tornar simples a solução atual, mas permitindo que modificações futuras possam acontecer ao invés de implementar algo complexo que jamais será utilizado. O \textit{feedback} significa que o cliente terá sempre uma parte do \textit{software} funcional para avaliar e que a equipe sempre terá uma resposta dos testes realizados e também do cliente para a melhoria da qualidade do produto. Para a prática dos três outros valores, o quarto se faz necessário: a coragem (SOARES, 2004).

\section{Princípios da metodologia ágil}
Além das quatro propostas descritas no “Manifesto Ágil para o desenvolvimento de \textit{Software} Ágil”, seus autores também propóem por este documento os 12 princípios fundamentais da metodologia ágil:
\begin{enumerate}
	\item “Nossa maior prioridade é satisfazer o cliente, através da entrega adiantada e contínua de \textit{software} de valor” (AGILE MANIFESTO, 2001). O Manifesto Ágil acredita que documentos de especificações e arquitetura de \textit{software}, por exemplo, são importantes para o desenvolvimento de software, mas o que importa ao cliente, de fato, é a funcionalidade do \textit{software} implementada, provando que a aplicação é um bom investimento para sua empresa.
	\item “Aceitar mudanças de requisitos, mesmo no fim do desenvolvimento. Processos ágeis se adequam a mudanças, para que o cliente possa tirar vantagens competitivas” (AGILE MANIFESTO, 2001). Sendo o fututo imprevisível, a abordagem ágil visa aceitar requisições de mudanças no \textit{software} mantendo a consciência das possíveis consequências.
	\item “Entregar \textit{software} funcionando com freqüencia, na escala de semanas até meses, com preferência aos períodos mais curtos” (AGILE MANIFESTO, 2001). Este princípio está atrelado à satisfação do cliente, pois a partir de um modelo de desenvolvimento iterativo e incremental é possível entregar \textit{software} (partes dele) em períodos curtos, entregar valor ao cliente em períodos curtos.
	\item “Pessoas relacionadas à negócios e desenvolvedores devem trabalhar em conjunto e diáriamente, durante todo o curso do projeto” (AGILE MANIFESTO, 2001). A metodoglia ágil trabalha a partir de requisitos com alto nível de abstração que, obviamente, não são suficientes para a realização completa do desenho e arquitetura da funcionalidade e posterior codificação. Assim, o intenso relacionamento entre os \textit{stakeholders} do projeto em questão.
	\item “Construir projetos ao redor de indivíduos motivados. Dando a eles o ambiente e suporte necessário, e confiar que farão seu trabalho” (AGILE MANIFESTO, 2001). Sendo o produto de \textit{software} produzido por seres humanos, são estas pessoas que determinam o sucesso ou o fracasso de um projeto de \textit{software}. Indivíduos motivados e confiantes podem colaborar para o sucesso do projeto.
	\item “O Método mais eficiente e eficaz de transmitir informações para, e por dentro de um time de desenvolvimento, é através de uma conversa cara a cara” (AGILE MANIFESTO, 2001). Documentos escritos podem ser ambíguos, mesmo que escritos da melhor maneira julgada pelo autor: conversas cara a cara devem resolver tal problema, visto que todas as dúvidas e confirmações sobre o assunto tratado são conversadas e não escritas.
	\item “\textit{Software} funcional é a medida primária de progresso” (AGILE MANIFESTO, 2001). Sem a abstração das características do sistema em um \textit{software} funcional não é possível constatar se o projeto terá sucesso.
	\item “Processos ágeis promovem um ambiente sustentável. Os patrocinadores, desenvolvedores e usuários, devem ser capazes de manter indefinidamente, passos constantes” (AGILE MANIFESTO, 2001). Passos constantes, no contexto de metodologia ágil, dizem respeito à jornada de trabalho da equipe: ao passo que a jornada foi definida, ela deve ser cumprida, evitando ao máximo horas extras e desgaste da equipe.
	\item “Contínua atenção à excelência técnica e bom design, aumenta a agilidade” (AGILE MANIFESTO, 2001). A abordagem ágil de desenvolvimento visa manter qualidade da arquitetura e do desenho do \textit{software} reconhecendo que isto é vital para a manutenção da agilidade do desenvolvimento. O trabalho de desenhar a arquitetura do \textit{software} ocorre em todas iterações do desenvolvimento.
	\item “Simplicidade: a arte de maximizar a quantidade de trabalho que não precisou ser feito” (AGILE MANIFESTO, 2001). Uma vez que a metodologia ágil aceita mudanças nos requisitos do \textit{software} facilmente, manter a simplicidade do \textit{software} colabora para que tais modificações sejam feitas de forma mais fácil e menos complexa.
	\item “As melhores arquiteturas, requisitos e \textit{designs} emergem de times auto-organizáveis” (AGILE MANIFESTO, 2001). Sendo arquiteturas, requisitos e desenhos advindos da criatividade humana, quando a interação entre membros da equipe é alto e as regras são poucas a criatividade é maior e os produtos são de maior qualidade.
	\item “Em intervalos regulares, o time reflete em como ficar mais efetivo, então, se ajustam e otimizam seu comportamento de acordo” (AGILE MANIFESTO, 2001). A equipe deve sempre procurar monitorar-se e molhorar seus próprios métodos e processo de desenvolvimento.

\end{enumerate}
