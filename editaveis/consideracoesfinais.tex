\chapter[Considerações Finais]{Considerações Finais}

De acordo com Rombach (1990), o desenho da arquitetura do \textit{software} tem mais influência na manutenibilidade do mesmo do que o desenho e implementação do algoritmo do \textit{software}.
Diversos experimentos foram realizados com o objetivo de averiguar a afirmação feita por Rombach. Em um de tais experimentos, foram elaboradas versões de um único sistema, cada versão com uma arquitetura diferente, mas utilizando-se as mesmas tecnologias. Dos resultados de qualidade obtidos, verificou-se o aumento da manutenibilidade do software quando é despendido um tempo considerável na arquitetura de \textit{software} (BRUSAMOLIN, 2004).

Visto que dentre os princípios da metodologia ágil existe uma preocupação contínua em conservar a auto-organização do time de desenvolvimento de modo a produzir melhores requisitos, arquiteturas e \textit{designs}, além de manter a atenção na construção de um bom \textit{design}, reconhecendo que a qualidade da arquitetura e do desenho de \textit{software} são importantes para a manutenção da agilidade do desenvolvimento e, consequentemente, da manutenibilidade do mesmo é possível afirmar que a arquitetura de \textit{software} é impactada de forma sempre positiva, mesmo que o incremento da arquitetura seja realizado em toda iteração.

Desta forma, podemos citar como impacto visível na manutenibiliade de \textit{software} quando este é construído no contexto ágil de desenvolvimento uma maior qualidade da arquitetura construída: a arquitetura é, geralmente, desenvolvida com base nos princípios SOLID aqui citados, uma vez que a aceitação de mudanças é facilitada e o modelo adotado é iterativo e incremental, ocorrendo entrega constante de \textit{software} ao cliente. Entregando \textit{software} (partes dele) em períodos curtos e incrementando a solução já existente, o desenvolvimento de software, após a primeira entrega, se assemelha à manutenção perfectiva e adaptativa, ao modificar ou incluir novas funcionalidades ao sistema de acordo com as requisições realizadas pelo cliente ou usuário e tais requisições podem ou não partir de uma mudança do ambiente operacional no qual parte do \textit{software} entregue está inserido.
