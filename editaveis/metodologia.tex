\chapter[Metodologia]{Metodologia}
Segundo Kitchenham (2007), a revisão sistemática de literatura consiste em uma técnica utilizada para “avaliação e interpretação de todas as pesquisas existentes disponíveis para uma questão de pesquisa, área temática ou fenômeno de interesse e possui como objetivo apresentar uma avaliação justa sobre o tópico de pesquisa utilizando uma metodologia confiável, rigorosa e passível de reprodução”.

Dado que uma das razões mais comuns para a execução de uma revisão sistemática é a necessidade de identificação de lacunas em uma presente pesquisa, que ajude a nomear novas áreas de estudo ou investigação e o objetivo da presente pesquisa é identificar tais gaps quando o assunto é impacto de mudanças de arquitetura de software na manutenibilidade do sistema, a metodologia adotada para a execução do presente trabalho consiste em uma revisão sistemática da literatura existente sobre os temas envolvidos.

\section{Classificação da pesquisa}
Uma pesquisa pode ser classificada de diversas formas e dentre estas formas, dentre tais formas existe a classificação quanto à abordagem da pesquisa, que pode ser qualitativa, quantitativa ou ambas (mista). Segundo Günther (2006), a pesquisa qualitativa é baseada na compreensão de conhecimentos e pesquisas já publicadas, tendo como objeto de estudo a “descoberta e construção de teorias” além de ser uma abordagem de pesquisa baseada em textos, ou seja, a coleta de dados para a pesquisa produz textos que são interpretados de acordo com o que foi escrito pelo autor da pesquisa. Partindo desta definição, pode-se afirmar que a presente pesquisa pode ser classificada como qualitativa, uma vez que os dados coletados não serão dados estatísticos ou tampouco será executada uma análise de dados de tal tipo.
\section{Identificação das fontes}
\subsection{Critérios para seleção de fontes}
Os critérios adotados para seleção de fontes foram:
\begin{itemize}
	\item Bibliotecas digitais.
	\item Artigos publicados em periódicos nacionais e internacionais.
	\item Artigos publicados em congressos, conferências e seminários nacionais e internacionais.
\end{itemize}

O idioma em que o artigo foi publicado é também um critério de seleção da fonte. Foram utilizados os idiomas Português e Inglês. O inglês foi utilizado pela quantidade considerável de artigos publicados neste idioma em congressos, conferências, seminários e periódicos internacionais e o português para consulta de trabalhos publicados no Brasil.
\subsection{Método de busca}
O método de busca utilizado foi determinado pelo mecanismo de busca fornecido pela fonte de pesquisa utilizando-se filtros de data e tipos de fonte (livros, artigos ou periódicos, por exemplo).
\subsection{String de Busca}
As strings de busca foram utilizadas individualmente afim de encontrar pesquisas que tratam os assuntos relacionados. As strings de busca utilizadas foram:
\begin{itemize}
	\item Processo de desenvolvimento de software = Software Development Process
	\item Metodologia ágil de desenvolvimento de software = Agile methodology of software development
	\item Manifesto Ágil = Agile Manifest
	\item Arquitetura de software = Software Architecture
	\item Mudanças na arquitetura de software = Modifications in software architecture
	\item Manutenção <or> manutenibilidade de software = Software Maintenance <or> maintenability
	\item Arquitetura <and> manutenibilidade de software = Software Maintenance <and> Architecture

\end{itemize}

\subsection{Lista de Fontes de Pesquisa}
Para busca de dados para esta pesquisa foi utilizado como ferramenta de busca o Google Acadêmico (https://scholar.google.com.br/) para que pudesse ser abrangido um número maior de fontes. O Google Acadêmico redireciona cada um dos trabalhos publicados para a sua respectiva “fonte raíz”. Deste modo, as bases de dados principais redirecionadas foram:
\begin{itemize}
	\item Safari (https://www.safaribooksonline.com/)
	\item IEEE Explore Digital Library (http:// http://ieeexplore.ieee.org/)
	\item Object Mentor (http://www.objectmentor.com/)
	\item Scielo (http://www.scielo.br/)
	\item Portal de periódicos CAPES (http://www.periodicos.capes.gov.br/)

\end{itemize}

\subsection{Seleção de Estudos}
\subsubsection{Critérios de inclusão de estudos} 
Os artigos foram incluídos na pesquisa em relação aos sequintes critérios:
\begin{itemize}
	\item Artigos e livros de natureza qualitativa que tratam de definições a respeito das principais fases do processo de desenvolvimento de software.
	\item Artigos e livros de natureza qualitativa que tratam de definições a respeito dos valores e princípios da metodologia ágil.
	\item Artigos e livros de natureza qualitativa de tratam de definições e princípios de arquitetura de software.
	\item Artigos e livros de natureza qualitativa que tratam de conceitos relacionados a manutenção de software.
	\item Artigos e livros devem estar disponíveis na web.
	\item Artigos e livros devem apresentar textos completos de estudos em formato eletrônico.
	\item Artigos e livros na área de computação e desenvolvimento de software.
\end{itemize}

A seleção e/ou exclusão dos artigos e/ou livros (ou parte deste) encontrados será baseada na leitura do Resumo, Introdução e das Considerações Finais, que deverão conter aspectos relacionados às variáveis de estudo desta pesquisa dentro do que é tratado pela área de Engenharia de Software. Apenas os artigos selecionados e incluídos deverão ser armazenados.